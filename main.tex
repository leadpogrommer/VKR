\documentclass[a4paper,14pt]{extarticle}
\usepackage[T2A]{fontenc}
\usepackage[utf8]{inputenc}
\usepackage[english,russian]{babel}

\usepackage[newfloat]{minted}
\usepackage[dvipsnames]{xcolor}
\definecolor{aliceblue}{rgb}{0.94, 0.97, 1.0}
\usepackage{tikz-uml}
\usepackage[normalem]{ulem}
\usepackage{csquotes}


\usepackage[right=15mm, left=25mm, top=15mm, bottom=20mm, nohead]{geometry}

\usepackage{amsmath}
\usepackage{amsthm}
\usepackage{enumitem}
\usepackage{mathtools}
\usepackage{cmap}
\usepackage{array}

\renewcommand{\baselinestretch}{1.2}
\newtheorem*{definition}{Опр}

\DeclareMathOperator*{\VarInstr}{VarInstr}
\DeclareMathOperator*{\offset}{offset}

\usepackage{caption}
%\newenvironment{code}{\captionsetup{type=listing}}{}
\SetupFloatingEnvironment{listing}{name=Листинг}
\newenvironment{longlisting}{\captionsetup{type=listing}}{}

\usepackage{hyperref}
\usepackage{indentfirst}

\usepackage[
	backend=biber, %подключение пакета biber (тоже нужен)
	bibstyle=gost-numeric, %подключение одного из четырех главных стилей biblatex-gost 
	citestyle=numeric-comp, %подключение стиля стиля (а вот!) 
	language=auto, %указание сортировки языков
	babel=other, %указание языков
	sorting=ntvy, %тип сортировки в библиографии
	doi=false, 
	eprint=false, 
	isbn=false, 
	dashed=false, 
	url=false 
]{biblatex}
\addbibresource{library.bib}
\usepackage{bibentry}

\graphicspath{{img/}}

%\DeclareFieldFormat{postnote}{#1} %убирает с. и p. 
%\renewcommand*{\multicitedelim}{\addsemicolon\space} % добавляет точку с запятой и пробел (; ) в перечислении
%\renewcommand*{\postnotedelim}{\addcolon\space}


\title{Эффективная генерация кода для платформы ARM32}
\author{}
\date{}

\begin{document}
	\begin{titlepage}
	\begin{center}	
		\footnotesize
		МИНИСТЕРСТВО ОБРАЗОВАНИЯ И НАУКИ РОССИЙСКОЙ ФЕДЕРАЦИИ 
		
		ФЕДЕРАЛЬНОЕ ГОСУДАРСТВЕННОЕ АВТОНОМНОЕ ОБРАЗОВАТЕЛЬНОЕ УЧРЕЖДЕНИЕ
		ВЫСШЕГО ОБРАЗОВАНИЯ
		
		«НОВОСИБИРСКИЙ НАЦИОНАЛЬНЫЙ ИССЛЕДОВАТЕЛЬСКИЙ ГОСУДАРСТВЕННЫЙ УНИВЕРСИТЕТ»
		
		(НОВОСИБИРСКИЙ ГОСУДАРСТВЕННЫЙ УНИВЕРСИТЕТ, НГУ)
		\vspace{0.25cm}
	\end{center}	
	\normalsize
	
	\noindent
	\begin{tabular}{l @{\hskip 1cm} l}
		Факультет &Информационных технологий \\
		Кафедра   &Систем информатики 
	\end{tabular}
	
	\vspace{0.5cm}
	
	\noindent
	\hskip 0.2cm Направление подготовки \hskip 0.3cm 09.03.01 Компьютерные науки и системотехника
	
	\vfill		
	
	\begin{center}
		\small	
		\textbf{ВЫПУСКНАЯ КВАЛИФИКАЦИОННАЯ РАБОТА БАКАЛАВРА}\\[4mm]
		\normalsize
		\uline{\hfill  Мерзляков Илья Алексеевич \hfill} 
		
		
		\normalsize
		Тема работы: 
		Разработка бэкенда LLVM для процессора CDM16
		%\bigskip		
	\end{center}
	\vfill
	
	\noindent
	\begin{tabular*}{\textwidth}{l @{\hskip 4cm} l}
		\textbf{\textquote{К защите допущена}}	& \textbf{Научный руководитель} \\
		& \\
		Заведующий кафедрой,		& Доцент ФФ НГУ \\ 
		д.ф.-м.н., профессор		&  \\ 
		& \\
		\uline{Лаврентьев М.М.}/\uline{\hspace{2cm}} & \uline{Иртегов Д.В.}/\uline{\hspace{2cm}} \\ [-1.1ex]
		\scriptsize (фамилия, И.,О.) \hspace{0.5cm} (подпись, МП) & \scriptsize (фамилия, И.,О.) \hspace{0.5cm} (подпись, МП) \\
		& \\ 
		«...»...............20...г.& «...»...............20...г. 			
		
	\end{tabular*}
	\vfill
	\hfill
	\begin{minipage}{0.5\textwidth}
		Дата защиты: «...».............20...г.
	\end{minipage}
	
	\vfill
	
	\begin{center}
		Новосибирск, 2024 г.
	\end{center}
\end{titlepage}
	
	

\tableofcontents

\pagebreak

\section{Введение}

На ФИТ НГУ на курсе «Цифровые платформы» студентами изучается учебный 8-битный процессор CdM8. Из-за ограничений его архитектуры (8-битное адресное пространство, позволяющее использовать только 256 байт памяти), CdM8 не подходит для реализации на нем сложных проектов. В 2023 году группой студентов 3 курса ФИТ НГУ на основе этого процессора был разработан 16-битный процессор CdM16, имеющий 16-битное адресное пространство и позволяющий реализовывать более сложные проекты. Однако, написание кода на языке ассемблера – трудоемкая задача, значительно увеличивающая время разработки, а реализаций высокоуровневых языков для CdM16 в настоящий момент не существует.

В связи с этим было решено создать компилятор языка Си для процессора CdM16.

\pagebreak
\section{Обзор вариантов создания компилятора Си под новую архитектуру}

Есть два способа получить компилятор Си под новую архитектуру: разработать этот компилятор с нуля, или добавить поддержку новой архитектуры в уже существующий компилятор. Для выбора оптимального подхода сначала рассмотрим стадии компиляции программ на Си.

% SRC: https://habr.com/ru/articles/478124/
Процесс компиляции программ на языке Си состоит из нескольких этапов:
\begin{itemize}
	\item Препроцессинг - обработка директив препроцессора (начинаются с '\#')
	\item Компиляция - преобразование отдельных единиц трансляции в ассемблер целевой платформы
	\item Ассемблирование - преобразование ассемблера в машинный код
	\item Компоновка - связывание результатов компиляции отдельных единиц трансляции в единый исполняемый файл
\end{itemize}

Последние два этапа уже реализованы авторами CdM16 в инструменте cocas, поэтому новому компилятору будет достаточно выполнять только препроцессинг и компиляцию. 

\subsection{Свой компилятор}

Компиляция программы на языке Си тоже состоит из нескольких этапов, и не все они зависят от целевой платформы:
препроцессинг, парсинг исходного кода, проверка типов и проведение некоторых оптимизаций производятся одинаково 
для всех архитектур, поэтому создание компилятора "с нуля" будет тратой ресурсов. 
Поэтому оптимальнее будет взять существующий компилятор, поддерживающий несколько архитектур, и добавить в него
поддержку CdM16.

\subsection{PPCI}

\href{https://github.com/windelbouwman/ppci}{PPCI (Pure Python Compiler Infrastructure)} - компилятор языка Си под различные архитектуры. Попытка добавить в него поддержку CdM16 ранее предпринималась создателем CdM16 Николаем Репиным. Из-за низкого качества кода PPCI и отсутствия поддержки (проект не обновляется уже 3 года) попытка завершилась неудачей.

\subsection{GCC}

TODO: Он старый, страшный, написан на Си.

\subsection{LLVM/CLANG}

TODO: Это круто, очень круто

\pagebreak
\section{Обзор архитектуры LLVM}

Изначально проект LLVM создавался как фреймворк для компиляции и \emph{оптимизации программ во время исполнения} с помощью анализа во время выполнения и JIT-компиляции\cite{LLVM:CGO04}. В настоящее время LLVM используется и развивается как инфраструктура для AOT компиляции.
% https://www.researchgate.net/figure/LLVM-Compiler-Development-architecture_fig2_334167635
\begin{figure}[!h]
	\begin{center}
		\includegraphics[width=\textwidth]{LLVM-Compiler-Development-architecture.png}
		\caption{Архитектура LLVM \cite{llvmpic}}
	\end{center}
\end{figure}

LLVM состоит из 3-х частей: фронтенда, транслирующего код на языке программирования в \emph{промежуточное представление} (IR), оптимизатора, производящего машинно-независимые оптимизации над IR, и бэкенда, транслирующего IR в инструкции целевой платформы. Так как фронтенд и оптимизатор не зависят от архитектуры, нам % кому нам я здесь одлин
понадобится разработать только бэкенд.

TODO: написать про IR % TODO: IR

\subsection{Архитектура бэкенда LLVM}
% https://llvm.org/docs/CodeGenerator.html#the-targetlowering-class
% https://llvm.org/docs/WritingAnLLVMBackend.html#instruction-relation-mapping

% TODO: операция - не самое подходящее слово
Бэкенд работает с IR в форме DAG (Direct Acyclic Graph) - графа, вершинами которого являются данные и операции над ними, а рёбрами - зависимость одних операций от результатов других. В основном вершины графа соответствуют операциям из IR, однако не весь IR автоматически представим в виде DAG. Некоторые операции, такие как вызов функций, получение их результата, получение значений параметров функции бэкенд должен 'спустить' (lower) в граф. Также не все операции и типы данных IR могут поддерживаться архитектурой (например, CdM16 не поддерживает 8-битные числа % TODO: в регистрах
), бэкенд должен 'легализовать' такие операции, т.е. заменить их несколькими поддерживаемыми операциями. % TODO: плохое предложение

После построения DAG происходит \emph{выбор} (selection) инструкций, этим занимается \emph{SelectionDAG}. Инструкции целевой платформы описываются на специальном языке \emph{tablegen}. Tablegen позволяет компактно задать мнемонику инструкции, её входные и выходные регистры и шаблон (pattern) из вершин DAG, которому она соответствует. Большинство инструкций автоматически выбираются на основе шаблонов, однако некоторые нужно выбирать вручную.

% TODO: как-то плоховато
После выбора инструкций бэкенд должен вывести их в текстовый ассемблерный файл. Части LLVM, ответственные за это, заточены под синтаксис GAS, который значительно отличается от ассемблера CdM16 \emph{cocas}, поэтому нам необходимо изменить логику вывода ассемблерного кода.


\pagebreak
\section{Обзор архитектуры CdM16}

% SRC: https://github.com/cdm-processors/cdm-devkit/blob/master/docs/cdm16/cdm16-overview.md
CdM16 - это 16-битный RISC % TODO: load store
 процессор, разработанный студентами НГУ для использования на курсе "Цифровые платформы". Основные технические характеристики:
\begin{itemize}
	\item 8 16-битных регистров общего назначения
	\item возможность адресовать 64 кБ адресного пространства
	\item арифметические инструкции принимают 3 операнда - 2 регистра-источника и регистр назначения
	
\end{itemize}

\pagebreak
\section{Реализация бэкенда LLVM для CdM16}
\subsection{Описание архитектуры}
\subsection{Арифметические операции}
\subsection{Соглашение о вызовах и стэк}
\subsection{Инструкции условного перехода}
\subsection{Вызов функций}
\subsection{Глобальные переменные}
\subsection{Линковка}
\subsection{Обработчики прерываний}

\pagebreak
\section{Заключение}

В результате работы был создан компилятор языка Си для платформы CdM16, поддерживающий значительное подмножество возможностей языка Си: создание функций, принимающий до 4 числовых аргументов, 8- и 16-битные переменные, циклы, ветвления, структуры, массивы, глобальные переменные, статические и внешние символы.

Пример проекта, написанного на Си под CdM16 - \href{https://github.com/leadpogrommer/llvm-project-cdm/tree/backend/cdm/llvm/test_cdm/life_multifile}{реализация игры "жизнь"}.

\subsection{Планы на будущее}

В будущем планируется реализовать:
\begin{itemize}
	\item программную реализацию 32-битной арифметики
	\item поддержку передачи аргументов на стеке
	\item поддержку функций с переменным количеством аргументов (vararg)
	\item минимальную стандартную библиотеку Си
\end{itemize}


\pagebreak
\printbibliography
\addcontentsline{toc}{section}{\refname}




	
\end{document}