\documentclass[a4paper,14pt]{extarticle}
\usepackage[T2A]{fontenc}
\usepackage[utf8]{inputenc}
\usepackage[english,russian]{babel}

\usepackage[newfloat]{minted}
\usepackage[dvipsnames]{xcolor}
\definecolor{aliceblue}{rgb}{0.94, 0.97, 1.0}
\usepackage{tikz-uml}
\usepackage[normalem]{ulem}
\usepackage{csquotes}


\usepackage[right=15mm, left=25mm, top=15mm, bottom=20mm, nohead]{geometry}

\usepackage{amsmath}
\usepackage{amsthm}
\usepackage{enumitem}
\usepackage{mathtools}
\usepackage{cmap}
\usepackage{array}

\renewcommand{\baselinestretch}{1.2}
\newtheorem*{definition}{Опр}

\DeclareMathOperator*{\VarInstr}{VarInstr}
\DeclareMathOperator*{\offset}{offset}

\usepackage{caption}
%\newenvironment{code}{\captionsetup{type=listing}}{}
\SetupFloatingEnvironment{listing}{name=Листинг}
\newenvironment{longlisting}{\captionsetup{type=listing}}{}

\usepackage{hyperref}
\usepackage{indentfirst}

\usepackage[
	backend=biber, %подключение пакета biber (тоже нужен)
	bibstyle=gost-numeric, %подключение одного из четырех главных стилей biblatex-gost 
	citestyle=numeric-comp, %подключение стиля стиля (а вот!) 
	language=auto, %указание сортировки языков
	babel=other, %указание языков
	sorting=ntvy, %тип сортировки в библиографии
	doi=false, 
	eprint=false, 
	isbn=false, 
	dashed=false, 
	url=false 
]{biblatex}
\addbibresource{library.bib}
\usepackage{bibentry}

%\DeclareFieldFormat{postnote}{#1} %убирает с. и p. 
%\renewcommand*{\multicitedelim}{\addsemicolon\space} % добавляет точку с запятой и пробел (; ) в перечислении
%\renewcommand*{\postnotedelim}{\addcolon\space}


\title{Эффективная генерация кода для платформы ARM32}
\author{}
\date{}

\begin{document}
	\begin{titlepage}
	\begin{center}	
		\footnotesize
		МИНИСТЕРСТВО ОБРАЗОВАНИЯ И НАУКИ РОССИЙСКОЙ ФЕДЕРАЦИИ 
		
		ФЕДЕРАЛЬНОЕ ГОСУДАРСТВЕННОЕ АВТОНОМНОЕ ОБРАЗОВАТЕЛЬНОЕ УЧРЕЖДЕНИЕ
		ВЫСШЕГО ОБРАЗОВАНИЯ
		
		«НОВОСИБИРСКИЙ НАЦИОНАЛЬНЫЙ ИССЛЕДОВАТЕЛЬСКИЙ ГОСУДАРСТВЕННЫЙ УНИВЕРСИТЕТ»
		
		(НОВОСИБИРСКИЙ ГОСУДАРСТВЕННЫЙ УНИВЕРСИТЕТ, НГУ)
		\vspace{0.25cm}
	\end{center}	
	\normalsize
	
	\noindent
	\begin{tabular}{l @{\hskip 1cm} l}
		Факультет &Информационных технологий \\
		Кафедра   &Систем информатики 
	\end{tabular}
	
	\vspace{0.5cm}
	
	\noindent
	\hskip 0.2cm Направление подготовки \hskip 0.3cm 09.03.01 Компьютерные науки и системотехника
	
	\vfill		
	
	\begin{center}
		\small	
		\textbf{ВЫПУСКНАЯ КВАЛИФИКАЦИОННАЯ РАБОТА БАКАЛАВРА}\\[4mm]
		\normalsize
		\uline{\hfill  Мерзляков Илья Алексеевич \hfill} 
		
		
		\normalsize
		Тема работы: 
		Разработка бэкенда LLVM для процессора CDM16
		%\bigskip		
	\end{center}
	\vfill
	
	\noindent
	\begin{tabular*}{\textwidth}{l @{\hskip 4cm} l}
		\textbf{\textquote{К защите допущена}}	& \textbf{Научный руководитель} \\
		& \\
		Заведующий кафедрой,		& Доцент ФФ НГУ \\ 
		д.ф.-м.н., профессор		&  \\ 
		& \\
		\uline{Лаврентьев М.М.}/\uline{\hspace{2cm}} & \uline{Иртегов Д.В.}/\uline{\hspace{2cm}} \\ [-1.1ex]
		\scriptsize (фамилия, И.,О.) \hspace{0.5cm} (подпись, МП) & \scriptsize (фамилия, И.,О.) \hspace{0.5cm} (подпись, МП) \\
		& \\ 
		«...»...............20...г.& «...»...............20...г. 			
		
	\end{tabular*}
	\vfill
	\hfill
	\begin{minipage}{0.5\textwidth}
		Дата защиты: «...».............20...г.
	\end{minipage}
	
	\vfill
	
	\begin{center}
		Новосибирск, 2024 г.
	\end{center}
\end{titlepage}
	
	

\tableofcontents

\pagebreak

\section{Введение}

На ФИТ НГУ на курсе «Цифровые платформы» студентами изучается учебный 8-битный процессор CdM8. Из-за ограничений его архитектуры (8-битное адресное пространство, позволяющее использовать только 256 байт памяти), CdM8 не подходит для реализации на нем сложных проектов. В 2023 году группой студентов 3 курса ФИТ НГУ на основе этого процессора был разработан 16-битный процессор CdM16, имеющий 16-битное адресное пространство и позволяющий реализовывать более сложные проекты. Однако, написание кода на языке ассемблера – трудоемкая задача, значительно увеличивающая время разработки, а реализаций высокоуровневых языков для CdM16 в настоящий момент не существует.

\pagebreak
\section{Обзор вариантов создания компилятора Си под новую архитектуру}
\subsection{LLVM/CLANG}
\subsection{GCC}
\subsection{TCC}
\subsection{Свой компилятор}

\pagebreak
\section{Обзор архитектуры LLVM}

\pagebreak
\section{Обзор архитектуры CdM16}

\pagebreak
\section{Реализация бэкенда LLVM для CdM16}
\subsection{Арифметические операции}
\subsection{Соглашение о вызовах и стэк}
\subsection{Инструкции условного перехода}
\subsection{Вызов функций}
\subsection{Глобальные переменные}
\subsection{Линковка}
\subsection{Обработчики прерывний}

\pagebreak
\section{Заключение}
\subsection{Планы на будущее}


\printbibliography
\addcontentsline{toc}{section}{\refname}




	
\end{document}